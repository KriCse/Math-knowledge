\chapter{Linear Maps}
\section{Vector spaces}
A vector space over a field \(F\) is a set \(V\) that is closed under vector addition (+) and scalar multiplication (\( \cdot \)). A vecotr space must fulfill following axioms:
\begin{align}
    \begin{split}
        v, w, u \in V \\
        \alpha, \beta \in F \\
        v + w = u  \\
        \alpha \cdot v = u \\
        (\alpha \cdot \beta) \cdot v = \alpha \cdot (\beta \cdot v)      \\
        \alpha (v + w) = \alpha \cdot v + \alpha \cdot w              \\
        (\alpha + \beta) \cdot w = \alpha \cdot v + \beta \cdot w \\
        1 \cdot v = w
    \end{split}
\end{align}
The vector addition \((V,+)\) form a commutative group.
\begin{align*}
    (v + w) + u = v + (w + z) \tag*{Associativity} \\
    v + 0 = V \tag*{Identity element: zero vector} \\
    v + w = 0 \tag*{Inverse element}               \\
    v + w  = w + v \tag*{Commutativity}            \\
\end{align*}
\( (F, +, \cdot)\) form a field.
\begin{align*}
    a, a^{-1}, b, c \in F                                                         \\
    (a + b) + c = a + (b + c) \tag*{Additive associativity}                       \\
    a + a^{-1} = 0 \tag*{Additive inverse}                                        \\
    a + 0 = a \tag*{Additive identity}                                            \\
    a + b = b + a \tag*{Additive commutativity}                                   \\
    (a \cdot b) \cdot c = a \cdot (b \cdot c)  \tag*{Mulitlicative associativity} \\
    a \cdot a^{-1} = 1, a^{-1} \neq 0 \tag*{Mulitlicative inverse}                \\
    a  \cdot 1 = a \tag*{Multiplicative identity}                                 \\
    a \cdot b = b \cdot a \tag*{Mulitplicative commutativity}                     \\
    a \cdot (b + c) = a\cdot b + \cdot c \tag*{Distibutivity}
\end{align*}
Example for fields: \((\mathbb{R}, +, \cdot)\) \( (\mathbb{C}, +, \cdot)\) \\
Example for non fields: \((\mathbb{N}, +, \cdot)\) \( (\mathbb{Z}, +, \cdot)\) \\
\subsection{Subspace}
Let \(V\) be a vector space ofer a field \(F\). A subspace \(W\) is a subset of \(V\) that also form a vector space over \(F\).
\begin{example}
    \(\mathbb{R}^2\) is a vector space over \(\mathbb{R}\).
    \begin{align*}
        W = \setb{c \cdot \begin{bmatrix}
                1 \\2
            \end{bmatrix}
        }{c \in \mathbb{R}} = \{ \dots \begin{bmatrix}
            -1 \\ -2
        \end{bmatrix}, \begin{bmatrix}
            1 \\2
        \end{bmatrix}, \begin{bmatrix}
            2 \\4
        \end{bmatrix},\begin{bmatrix}
            2.4 \\ 4.8
        \end{bmatrix} \dots
        \}
    \end{align*}
\end{example}
\subsubsection{Direct sum}
Let \(V\) be a vector space ofer a field \(F\)  and \(W\) and \(U\) subspaces of \(V\)
When:
\begin{align}
    W \cap U & = \{ 0 \} \\
    W \cup U & = V
\end{align}
Then \(U\) and \(W\) are a direct sum of \(V\). It is denoted by
\begin{equation}
    V = U \oplus W
\end{equation}
\subsection{Span, (lineare Hülle)}
Let \(V\) be a vector space ofer a field \(F\) and \(S\) a finite subset of \(V\) wiht length \(n\). The span of \(S\) is the set of vectors that can be created by linear combinations with the vectors in \(S\).
\begin{equation}
    span(S) = \setb{
        \sum_{i=1}^{n} a_i \cdot s_i
    }{n \in \mathbb{N}, a_i \in F, s_i \in S}
\end{equation}
\begin{example}
    \begin{align*}
        S        & = \{
        \begin{bmatrix}
            1 \\ 0 \\ 0
        \end{bmatrix},
        \begin{bmatrix}
            0 \\ 1 \\ 0
        \end{bmatrix}
        \}                   \\
        \vspan S & =  \setb{
            a_1
            \begin{bmatrix}
                1 \\ 0 \\ 0
            \end{bmatrix} + a_2
            \begin{bmatrix}
                0 \\ 1 \\ 0
            \end{bmatrix}
            \
        }{a_1, a_2 \in \mathbb{R}}
    \end{align*}
\end{example}
\subsubsection{Spanning set}
Let \(V\) be a vector space ofer a field \(F\) and \(S\) a finite subset of \(V\). \(S\) is a spanning set of if
\begin{equation}
    \vspan S = V
\end{equation}
\begin{example}
    Let \(V\) be \(\mathbb{R}^2\) over the field \(\mathbb{R}\). Following subsets are spanning set of \(V\):
    \begin{align*}
        S_1 & = \{ \begin{bmatrix}
            1 \\ 0
        \end{bmatrix},
        \begin{bmatrix} 0 \\ 1 \end{bmatrix}
        \}                                     \\
        S_2 & = \{ \begin{bmatrix}
            3 \\ 0
        \end{bmatrix},
        \begin{bmatrix} 0 \\ 2 \end{bmatrix}
        \}                                     \\
        S_3 & = \{ \begin{bmatrix}
            3 \\ 0
        \end{bmatrix},
        \begin{bmatrix} 0 \\ 2 \end{bmatrix},
        \begin{bmatrix} 1 \\ 2 \end{bmatrix}
        \}
    \end{align*}
\end{example}
\subsection{Base}
Let \(V\) be a vector space ofer a field \(F\) and \(B\) a spanning set of \(V\). If the elements of \(B\) are lineary independent then \(B\) is called a basis.
The coefficients of the linear combination are referred to as components or coordinates of the vector with respect to \(B\).
The elements of \(B\) are called basis vectors.
\begin{example}
    Let \(V\) be \(\mathbb{R}^2\) over the field \(\mathbb{R}\). Following subsets are spanning set of \(V\):
    \begin{align*}
        B_1 & = \{ \begin{bmatrix}
            1 \\ 0
        \end{bmatrix},
        \begin{bmatrix} 0 \\ 1 \end{bmatrix}
        \}                                     \\
        B_2 & = \{ \begin{bmatrix}
            3 \\ 0
        \end{bmatrix},
        \begin{bmatrix} 0 \\ 2 \end{bmatrix}
        \}                                     \\
    \end{align*}
    In previous example \(S_3\) is not a valid base, because its elements are lineary dependent.
\end{example}
\subsubsection{Standard base}
A base \(B\) is called a Standard base if the vecotors of \(B\) are all zero, except one that equals 1. The vectors
of the standard base are called unit vectors.
\begin{example}
    The standard base for \(\mathbb{R}^n\) is
    \begin{align*}
        \hat{i} & = \begin{bmatrix}
            1 \\ 0 \\ 0
        \end{bmatrix}, \hat{j} = \begin{bmatrix}
            0 \\ 1 \\ 0
        \end{bmatrix}, \hat{k} = \begin{bmatrix}
            0 \\ 0 \\ 1
        \end{bmatrix} \\
        B       & = \{ \hat{i}, \hat{j}, \hat{k} \}
    \end{align*}
    A vector \(v\) expressed in the standard basis \(B\).
    \begin{align*}
        v = \begin{bmatrix}
            4 \\ 5 \\ 6
        \end{bmatrix}= 4 \hat{i} + 5  \hat{j} + 6  \hat{k}
    \end{align*}
\end{example}
\subsection{Dimension}\label{dimension}
The dimension \( \dim \) of a vector space is the size of its base \(B\). The dimension
is equal to the rank (see \ref{rank}) of the tranformation matrix.
\begin{example}
    The vector space \(\mathbb{R}^n\)
    \begin{align*}
        \dim \mathbb{R}^n = n
    \end{align*}
\end{example}
\begin{example}
    \begin{align*}
        V      & = \setb{
            c_1
            \begin{bmatrix}
                1 \\ 0 \\ 0 \\ 0
            \end{bmatrix},
            c_2
            \begin{bmatrix}
                0 \\ 1 \\ 0 \\ 0
            \end{bmatrix}
        }{c_1, c_2 \in \mathbb{R}} \\
        \dim V & = 2
    \end{align*}
\end{example}
\begin{example}
    \begin{align*}
        V      & = \setb{
            c_1
            \begin{bmatrix}
                1 \\ 1 \\ 0 \\ 0
            \end{bmatrix},
            c_2
            \begin{bmatrix}
                0 \\ 1 \\ 0 \\ 0
            \end{bmatrix}
        }{c_1, c_2 \in \mathbb{R}} \\
        \dim V & = 1
    \end{align*}
\end{example}
\begin{example}
    The only vector space with dimension 0 is where V contains only the zero vector.
    \begin{align*}
        \dim \{ \begin{bmatrix} 0 \\ 0 \\ 0 \end{bmatrix}\} = 0
    \end{align*}

\end{example}
section{Linear maps}
Let \(V\) \(W\) be vector spaces over the same field \(F\).
A function \(f: V \rightarrow W\) is said to be a linear map if for any two vectors \(v, u \in V\) and any scalar \(c \in F\) the following two conditions are satisfied:
\begin{align}
    f(u + v)     & = f(u) + f(v) \tag{Additvity} \\
    f(c \cdot u) & = c \cdot u \tag{Homogenity}
\end{align}
\subsection{Transformation matrix}
Each linear transformation can be represented as a matrix vector multiplication.
\begin{align*}
    f      & : W \rightarrow V  \\
    \dim W & = n, \dim V = m    \\
    f(x)   & = A^{m \times n} x \\
\end{align*}
\begin{example}
    \begin{align*}
        f(x)                                       &
        = \begin{bmatrix}
            1 & 2 & 3 \\ 5 & 7 & 9
        \end{bmatrix} \begin{bmatrix}
            x_1 \\ x_2 \\ x_3
        \end{bmatrix}                   \\
        f \left(\begin{bmatrix}
            5 \\ 6 \\ 7
        \end{bmatrix} \right) & = \begin{bmatrix}
            38 \\ 130
        \end{bmatrix}
    \end{align*}
\end{example}
\subsubsection{Composition}
If there are two linear maps \(f, g\) wiht tranformation matrices:
\begin{align*}
    f : V \rightarrow W  = Ax \\
    g : U \rightarrow V = Bx
\end{align*}
then the compostition is:
\begin{align*}
    h : U \rightarrow W = f \circ g \\
    h(x) = A(Bx) = (A \cdot B)x
\end{align*}
\begin{example}
    \begin{align*}
        f(x)                                        & = \begin{bmatrix}
            1 & 2 & 3 \\ 5 & 7 & 9
        \end{bmatrix} \cdot x                                                                             \\
        g(x)                                        & = \begin{bmatrix}
            -1 & -2 \\ -7 & -9 \\ 13 & 17
        \end{bmatrix} \cdot x                                                                             \\
        h(x)                                        & = f \circ g = \begin{bmatrix}
            1 & 2 & 3 \\ 5 & 7 & 9
        \end{bmatrix} \begin{bmatrix}
            -1 & -2 \\ -7 & -9 \\ 13 & 17
        \end{bmatrix} \cdot x = \begin{bmatrix}
            24 & 31 \\ 64  & 80
        \end{bmatrix} \cdot x \\
        g \left( \begin{bmatrix}
            3 \\ 4
        \end{bmatrix} \right) & = \begin{bmatrix}
            -11 \\ -57 \\ 107
        \end{bmatrix}, f \left( \begin{bmatrix}
            -11 \\ -57 \\ 107
        \end{bmatrix} \right) = \begin{bmatrix}
            196 \\ 509
        \end{bmatrix}           \\
        h(\begin{bmatrix}
            3 \\ 4
        \end{bmatrix})               & = \begin{bmatrix}
            196 \\ 509
        \end{bmatrix}
    \end{align*}
\end{example}
\subsection{Image (Bild)}
The image \(f^{\rightarrow}\) of a tranformation \(L: V \rightarrow W \) is the set of vectors that the tranformation can produce.
\begin{equation}
    f^{\rightarrow} (L)  = \setb{L(x)}{x \in V}
\end{equation}
The image is the columnspan of the tranformation matrix. The dimenson of the image ist called \textbf{rank}, and is the
same as the rank of the transformation matrix.
\begin{example}
    \begin{align*}
        L(x)                & = \begin{bmatrix}
            1 & 0 \\
            0 & 1 \\
        \end{bmatrix}x \\
        f^{\rightarrow} (L) & = \mathbb{R}^2
    \end{align*}
\end{example}
\begin{example}
    \begin{align*}
        L(x)                & = \begin{bmatrix}
            1 & 0 \\
            2 & 0 \\
            1 & 1 \\
            0 & 2 \\
        \end{bmatrix}x      \\
        f^{\rightarrow} (L) & = \setb{\begin{bmatrix}
                c_1 \\ 2c_1 \\ c_1 + c_2 \\ 2 c_2
            \end{bmatrix}
        }{c_1, c_2 \in \mathbb{R}}
    \end{align*}
\end{example}
\subsection{Kernel, Null Space (Kern)}\label{kernel}
The kernel of a linear map \(L: V \rightarrow W \) is the linear subspace of the domain of the map which is mapped to the zero vector.
\begin{equation}
    \ker L = \{ v \in V | L(v) = 0 \} \\
\end{equation}

The vecots of the kernel are the set of vectors that yield the zero vector after multiplication with the tranformation matrx.
\begin{align*}
    L  & : Ax = y                          \\
    x' & \in \ker L  \textbf{ if } Ax' = 0
\end{align*}
The kernel forms a subspace of \( V \):
\begin{align*}
    v, u \in \ker L, \alpha \in \mathbb{F} \\
    \alpha v \in \ker L                    \\
    v + u \in \ker L                       \\
\end{align*}
The dimension of the kernel is called the \textbf{nullity}.
\begin{example}
    \begin{align*}
        L & : Ax = y                        \\
        A & =  \begin{bmatrix}
            1 & -1, & 0 \\
            0 & -2  & 4 \\
        \end{bmatrix} & \\
    \end{align*}
    To calculate \( \ker A\) simply set \(y\) to the zero vector and solve for \(x\).
    \begin{align*}
        \begin{bmatrix}
            1 & -1, & 0 \\
            0 & -2  & 4 \\
        \end{bmatrix} \begin{bmatrix}
            x_1 \\
            x_2 \\
            x_3 \\
        \end{bmatrix} = \begin{bmatrix}
            0 \\ 0 \\ 0
        \end{bmatrix} \\
        x_1 - x_2       = 0 \rightarrow x_1 = x_2                                          \\
        -2 x_2 + 4 x_3  = 0 \rightarrow x_2 = 2 x_3                                        \\
    \end{align*}
    The kernel is:
    \begin{align*}
        \ker L & = \setb{
            c \cdot \begin{bmatrix}
                2 \\ 2 \\ 1
            \end{bmatrix}
        }{c \in \mathbb{C} }
    \end{align*}
    A concrete example:
    \begin{align*}
        \begin{bmatrix}
            4 \\ 4 \\ 2
        \end{bmatrix}                 & \in \ker L                                                \\
        L \left(\begin{bmatrix}
            4 \\ 4 \\ 2
        \end{bmatrix} \right) & = \begin{bmatrix}
            1 \cdot 4  -1 \cdot 4  + 0 \cdot 2 \\
            0 \cdot 4  -2 \cdot 4 + 4 \cdot 2  \\
        \end{bmatrix} = \begin{bmatrix}
            0 \\ 0
        \end{bmatrix} \\
    \end{align*}
    \begin{example}

    \end{example}
    \section{Rank-Nullity theorem}
    If \(L : V \rightarrow W\) is a linear tranformation then it.
    \begin{equation}
        \rank L + \nullity T = \dim (image T) + \dim(\ker(T)) = \dim(V)
    \end{equation}
    \begin{example}
        \begin{align*}
            L          & : \begin{bmatrix}
                1 & 0 \\ 0 & 1
            \end{bmatrix} x = y \\
            image T    & = \mathbb{R}^2                     \\
            \rank L    & = 2                                \\
            \ker T     & = \{ \begin{bmatrix}
                0 \\ 0
            \end{bmatrix} \} \\
            \nullity L & = 0                                \\
            \dim V     & = \rank L + \nullity T = 2 +0 = 2
        \end{align*}
    \end{example}
    \begin{example}
        \begin{align*}
            L      & : \begin{bmatrix}
                1 & 2 & 4 \\ 2 & 4 & 8
            \end{bmatrix} x = y \\
            imag L & = \setb{
                \begin{bmatrix}
                    c \\ 2 c
                \end{bmatrix}
            }{c \in \mathbb{R}}
        \end{align*}
        \begin{align*}
            \rank L = 1 \\
            \ker L = \setb{
                \begin{bmatrix}
                    -2 c_1 - 4 c_2 \\ c_1 \\  c_2
                \end{bmatrix}
            }{ c_1, c_2 \in \mathbb{R}}
            \nullity A = 2
        \end{align*}
    \end{example}
\end{example}