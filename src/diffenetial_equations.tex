\chapter{Differential Equation}
\section{Homogeneous linear first-order}
The homogeneous linear first-order differential equations have the form:
\begin{align*}
   f'(t) + p(t)f(t) = 0
\end{align*}
Homogeneous is because one side of the equation is zero.
You can rewrite the expression above to have \(f(x)\) separated
\begin{gather}
   f'(t)  = -p(t)f(t) \\
   f'(t) \frac{1}{f(t)} = -p(t) \label{homodiff}
\end{gather}
Now all the \(f(t)\) terms are on the left hand side.

Note the following differentiation:
\begin{gather}
   (\ln f(t))' = \frac{1}{f(t)} f'(t)
\end{gather}
It can be used to help integrate \ref{homodiff}.
\begin{gather*}
   \int f'(t) \frac{1}{f(t)} = \int -p(t) \\
   \ln(|f(t)|) + C_1  =- P(t) + C_2 \\
   \ln(|f(t)|)  = -P(t) + \hat{C}
\end{gather*}
You can combine \(C_1\) and \(C_2\) to a single constant \(\hat{C}\), because they
both are constants.
Since the domain of \( \ln \) is \((0, \inf]\) you have to take the absolute value of \(f(t)\).
To get rid of \(\ln\) raise both side to \( e \). To compensate for the absolute value
you have to take \(\pm\) of \(e\).
\begin{gather}
   |f(t)| = e^{-P(t) + \hat{C}} \\
   f(t) = \pm e^{-P(t) + \hat{C}} \\
   f(t) =  e^{-P(t)} C \label{homogeneral}
\end{gather}
The expression \(e^{\hat{C}}\) is a constant, so it can be replaced by \(C\), which constant
be \(\pm\).

The expression \ref{homogeneral} is the general solution for homogeneous first order
linear differential equations. Any linear combination of the general is a valid solution
to the differential equation.
\subsubsection{Notation}
The notation of differential equations can be simplified by:
\begin{gather*}
   f(t) = y\\
   f(t)' = y'\\
\end{gather*}
\begin{example}
   \begin{gather*}
      y' + sin(x + 2)y = 0
   \end{gather*}
   The general solution is:
   \begin{gather*}
      p(t) = sin(x + 2) \\
      P(t) = - cos(x + 2)\\
      y =  e^{-p(t)} C \\
      y =  e^{cos(x + 2)} C \\
   \end{gather*}
   Following functions are valid solutions to the homogeneous equation:
   \begin{gather*}
      y =  2e^{cos(x + 2)} \\
      y =  2e^{cos(x + 2)} +  3e^{cos(x + 2)} \\
      y = 0 \\
   \end{gather*}
\end{example}
\subsubsection{Non-homogenous}
\begin{gather}\label{inhomo}
   y' + p(t)y = s(t)
\end{gather}
Recall the product rule which states
\begin{gather}
   (f \cdot g) = f' g + f g'
\end{gather}
If we multiply \ref{inhomo} by an unknown function \(\mu(t)\) called the \textbf{integrating factor} we get the following expression:
\begin{gather}\label{intfactor}
   \mu(t) y' + \mu(t) p(t)y = \mu(t) s(t)
\end{gather}
we have an expression that looks like the product rule, if we suppose that
\begin{gather}\label{intfactorcondition}
   \mu(t)' =  \mu(t) p(t) \\
   \mu(t) = \int \mu(t) p(t)
\end{gather}
Integrating \ref{intfactor} by using the product formula in reverse we get:
\begin{gather}
   \int \mu(t) \cdot y' + \mu(t)\cdot p(t) \cdot y  dt = \int \mu(t) \cdot s(t) dt \\
   \mu (t) y = \int \mu(td) s(t) dt \\
   y = \frac{1}{\mu(t)}\int \mu(t) s(t)  \label{inhomosol}
\end{gather}
The function that satisfies \ref{intfactorcondition} is
\begin{gather}
   \mu(t) = C(t) e^{P(t)}\\
   \mu(t)' = C(t)' e^{P(t)} + C(t) \cdot p(t) \cdot e^{P(t)}
\end{gather}
We can chose \(C(t) = 1\)
\begin{gather}
   \mu(t)  = e^{P(t)} \label{mufunction}\\
   \mu(t)' = p(t)e^{P(t)} = p(t)\mu(t)\\
\end{gather}
Setting \ref{mufunction} in \ref{inhomosol} we get the solution for the differential equation.
\begin{equation}
   y = e^{-P(t)}\int e^{P(t)} s(t)dt
\end{equation}
In general integrating the expression above yield the following expression
\begin{equation}
   y = y_p + y_h
\end{equation}
Where \(y_h\) is the solution to the homogenous equation \(y' + p(t)y = 0\). The term
\(y_p\) is called a particular solution and it is one of the solutions to the nonhomogeneous equation.
\begin{example}
   \begin{gather*}
      y' + 2y = t \\
      p(t) = 2\\
      P(t) = 2t\\
      s(t) = t \\
      y = e^{-2t}\int e^{2t} t dt\\
      y = e^{-2t}\left( \frac{1}{2} e^{2t} \cdot t - \frac{1}{4}e^{2t} + C \right)\\
      y = \frac{1}{2} t- \frac{1}{4} + C e^{-2t}
   \end{gather*}
   We see that  \(Ce^{-2t}\) is the solution to the homogeneous equation.
   \(y_p = \frac{1}{2} t- \frac{1}{4}\) is one of the solution to the nonhomogeneous equation.
\end{example}
\begin{matlab}
   \apilink{dsolve}{https://www.mathworks.com/help/symbolic/dsolve.html}
   \begin{lstlisting}
   >> syms y(t)
   >> eqn = diff(y, t) + 2*y == t
   >> dsolve(eqn)
   ans =
      t/2 + (C1*exp(-2*t))/4 - 1/4
  \end{lstlisting}
   Note: C1 / 4 is still a constant C.
\end{matlab}
\section{Linear first-oder system of differential equations}
\subsubsection{The exponential map}
The exponential map is defined by
\begin{equation}
   P_n(t) = e^{At}
\end{equation}
where \(A\) is a square matrix with dimension \(n \times n\) (see also \ref{sec:matrixexponent}).
If a is diagonalizabe then:
\begin{equation}
   P_n(t) = Ue^{Dt} U^{-1}
\end{equation}
The derivate is given by:
\begin{equation}
   P_n(t) \dt = A e^{At}
\end{equation}
\subsubsection{System of differential equations}
A system off differential equations contains a set of unkown function \(x_1(t), x_2(t) \cdots x_n(t) \) denoted by \(x_1, x_2 \cdots x_n\). The derivate of a funciton \(x_i\) (with respext to to \(t\)) is denoted by
\(x_i'\).
\subsection{Homogenous case}
A homogenous linear first-order system has the form:
\begin{equation*}
   \begin{split}
      x_1' &+ p_{11}x_1 + p_{12} x_2 \cdots p_{1n}x_{1n} = 0 \\
      x_2' &+ p_{21}x_1 + p_{22} x_2 \cdots p_{1n}x_{2n} = 0 \\
      &\vdots \\
      x_3' &+ p_{n1}x_1 + p_{n2} x_2 \cdots p_{1n}x_{nn}= 0 \\
   \end{split}
\end{equation*}
The terms \(p_{ij}\) are called coefficient functions (so written out the are like \(p_{ij}(t)\)) they only depend on t.
The derivates can be expressed as a column vector:
\begin{equation*}
   \bm{x'} = \begin{bmatrix}
      x_1' \\ x_2' \\ \vdots \\ x_n'
   \end{bmatrix}
\end{equation*}
The coefficient functions can be expressed as a square matrix \(A^{n \times n}\)
\begin{equation}
   \bm{p} = \begin{bmatrix}
      p_{11} & p_{12} & \cdots & p_{1n} \\
      p_{21} & p_{22} & \cdots & p_{2n} \\
      \vdots &        &        &        \\
      p_{n1} & p_{n2} & \cdots & p_{nn}
   \end{bmatrix}
\end{equation}
The functions can be written as a column vector
\begin{equation}
   \bm{x} = \begin{bmatrix}
      x_1 & x_2 & \cdots & x_n
   \end{bmatrix}
\end{equation}
The wholse system can be written in matrix form:
\begin{equation}
   \bm{x'} = \bm{p} \bm{x}
\end{equation}
The solution is given by (same formula as \ref{homogeneral} applied to matrices):
\begin{equation}
   \bm{x} = M_p(t) \bm{c}
\end{equation}
where \(\bm{c}\) is a column vector of constants.
\begin{example}
   \begin{gather*}
      x_1' = 1 x_1 + 4 x_2 \\
      x_2' = 3x_2 + 2 x_2 \\
   \end{gather*}
   Which can be rewritten as:
   \begin{gather*}
      \bm{x'} = \bm{p} \bm{x} \\
      \begin{bmatrix}
         x_1' \\ x_2'
      \end{bmatrix} = \begin{bmatrix}
         1 & 4 \\ 2 & 4
      \end{bmatrix} \begin{bmatrix}
         x_1 \\ x_2
      \end{bmatrix}
   \end{gather*}
   The eigenvalues and eigenvectors are:
   \begin{gather*}
      \lambda_1 = -2, v_1 = \begin{bmatrix}
         \frac{-4}{3} \\ 1
      \end{bmatrix}
      \lambda_2 = 5, v_1 = \begin{bmatrix}
         1 \\ 1
      \end{bmatrix}
   \end{gather*}
   Calculating \(M_p(t)\)
   \begin{gather*}
      M_p(t) = \begin{bmatrix}
         \frac{-4}{3} & 1 \\ 1 & 1
      \end{bmatrix}
      \begin{bmatrix}
         e^{-2t} & 0 \\ 0 & e^{5t}
      \end{bmatrix} \frac{1}{7}\begin{bmatrix}
         -3 & 3 \\ 4 & 4
      \end{bmatrix} = \frac{1}{7 }\begin{bmatrix}
         4 e^{-2t} + 3e^{5t}  & 4e^{5t} - 4e^{-2t}   \\
         -3 e^{-2t} + 3e^{5t} & 3 e^{-2t} + 4 e^{5t}
      \end{bmatrix}
   \end{gather*}
   The solution is (note the factor \(\frac{1}{7}\) can be ignored because \(\frac{c}{7}\) ist still a constant):
   \begin{gather}
      \bm{x} =  M_p(t) \bm{c} \\
      x_1(t) = c_1 \left( 4 e^{-2t} + 3e^{5t} \right) + c_2 \left(4e^{5t} - 4e^{-2t} \right) \\
      x_2(t) = c_1 \left(-3 e^{-2t} + 3e^{5t} \right) + c_2 \left(3 e^{-2t} + 4 e^{5t} \right)
   \end{gather}
   The constans can be simplified by setting \(c_a = (3c_1 + 4c_2)\) and \( c_b = 3(c_2 - c_1) \):
   \begin{align*}
      x_1(t) & = c_1 4 e^{-2t} + c_1 3e^{5t} + c_2 4 e^{5t} - c_2 4e^{-2t} = -4(c_2 - c_1)e^{-2t}  +  (3c_1 + 4c_2)e^{5t}    \\
             & = c_a e^{5t} - \frac{4}{3} c_b e^{-2t}                                                                        \\
      x_2(t) & = c_1 (-3) e^{-2t} + c_1 3e^{5t}  + c_2 3 e^{-2t} + c_2 4 e^{5t} = 3(c_2 - c_1) e^{-2t} + (3c_1 + 4c_2)e^{5t} \\
             & = c_b e^{-2t} + c_a e^{5t}
   \end{align*}
   \begin{matlab}
      \begin{lstlisting}
         >> syms x1(t) x2(t)
         >> ode1 = diff(x1) == 1*x1 + 4*x2
         >> ode2 = diff(x2) == 3*x1 + 2 *x2
         >> S = dsolve([ode1 ; ode2])
         >> S.x1
            ans = 
            C1*exp(5*t) - (4*C2*exp(-2*t))/3
         >> S.x2
            ans = 
            C2*exp(-2*t) + C1*exp(5*t)
      \end{lstlisting}
   \end{matlab}
\end{example}
A more simplified solution is:
\begin{equation}
   \bm{x} = \sum_{i=1}^{n} c_i \bm{v_i} e^{\lambda_i t}
\end{equation}