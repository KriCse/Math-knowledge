\chapter{Differential Equation}
\section{Homogeneous linear first-order}
The homogeneous linear first-order differential equations have the form:
\begin{align*}
   f'(t) + p(t)f(t) = 0
\end{align*}
Homogeneous is because one side of the equation is zero. 
You can rewrite the expression above to have \(f(x)\) separated
\begin{gather}
   f'(t)  = -p(t)f(t) \\
f'(t) \frac{1}{f(t)} = -p(t) \label{homodiff}
\end{gather}
Now all the \(f(t)\) terms are on the left hand side.

Note the following differentiation:
\begin{gather}
    (\ln f(t))' = \frac{1}{f(t)} f'(t)
\end{gather}
It can be used to help integrate \ref{homodiff}.
\begin{gather*}
\int f'(t) \frac{1}{f(t)} = \int -p(t) \\
\ln(|f(t)|) + C_1  =- P(t) + C_2 \\
\ln(|f(t)|)  = -P(t) + \hat{C}
\end{gather*}
You can combine \(C_1\) and \(C_2\) to a single constant \(\hat{C}\), because they
both are constants. 
Since the domain of \( \ln \) is \((0, \inf]\) you have to take the absolute value of \(f(t)\).
To get rid of \(\ln\) raise both side to \( e \). To compensate for the absolute value
you have to take \(\pm\) of \(e\).
\begin{gather}
   |f(t)| = e^{-P(t) + \hat{C}} \\
   f(t) = \pm e^{-P(t) + \hat{C}} \\
   f(t) =  e^{-P(t)} C \label{homogeneral}
\end{gather} 
The expression \(e^{\hat{C}}\) is a constant, so it can be replaced by \(C\), which constant
be \(\pm\). 

The expression \ref{homogeneral} is the general solution for homogeneous first order
linear differential equations. Any linear combination of the general is a valid solution 
to the differential equation.
\subsubsection{Notation}
The notation of differential equations can be simplified by: 
\begin{gather*}
    f(t) = y\\
    f(t)' = y'\\
\end{gather*}
\begin{example}
    \begin{gather*}
       y' + sin(x + 2)y = 0 
    \end{gather*}
    The general solution is: 
    \begin{gather*}
       p(t) = sin(x + 2) \\
       P(t) = - cos(x + 2)\\
       y =  e^{-p(t)} C \\
       y =  e^{cos(x + 2)} C \\
    \end{gather*} 
    Following functions are valid solutions to the homogeneous equation: 
    \begin{gather*}
       y =  2e^{cos(x + 2)} \\
       y =  2e^{cos(x + 2)} +  3e^{cos(x + 2)} \\
       y = 0 \\ 
    \end{gather*}
\end{example}
\subsubsection{Non-homogenous}
\begin{gather}\label{inhomo}
    y' + p(t)y = s(t) 
\end{gather}
Recall the product rule which states
\begin{gather}
   (f \cdot g) = f' g + f g'
\end{gather}
If we multiply \ref{inhomo} by an unknown function \(\mu(t)\) called the \textbf{integrating factor} we get the following expression:
\begin{gather}\label{intfactor}
  \mu(t) y' + \mu(t) p(t)y = \mu(t) s(t) 
\end{gather}
we have an expression that looks like the product rule, if we suppose that 
\begin{gather}\label{intfactorcondition}
  \mu(t)' =  \mu(t) p(t) \\
  \mu(t) = \int \mu(t) p(t)
\end{gather}
Integrating \ref{intfactor} by using the product formula in reverse we get:
\begin{gather}
 \int \mu(t) \cdot y' + \mu(t)\cdot p(t) \cdot y  dt = \int \mu(t) \cdot s(t) dt \\
 \mu (t) y = \int \mu(td) s(t) dt \\
  y = \frac{1}{\mu(t)}\int \mu(t) s(t)  \label{inhomosol}
\end{gather}
The function that satisfies \ref{intfactorcondition} is 
\begin{gather}
  \mu(t) = C(t) e^{P(t)}\\
  \mu(t)' = C(t)' e^{P(t)} + C(t) \cdot p(t) \cdot e^{P(t)}
\end{gather}
We can chose \(C(t) = 1\)
\begin{gather}
   \mu(t)  = e^{P(t)} \label{mufunction}\\ 
\mu(t)' = p(t)e^{P(t)} = p(t)\mu(t)\\
\end{gather}
Setting \ref{mufunction} in \ref{inhomosol} we get the solution for the differential equation.
\begin{equation}
  y = e^{-P(t)}\int e^{P(t)} s(t)dt
\end{equation}
In general integrating the expression above yield the following expression
\begin{equation}
  y = y_p + y_h
\end{equation}
Where \(y_h\) is the solution to the homogenous equation \(y' + p(t)y = 0\). The term
\(y_p\) is called a particular solution and it is one of the solutions to the nonhomogeneous equation.  
\begin{example}
   \begin{gather*}
  y' + 2y = t \\ 
  p(t) = 2\\ 
  P(t) = 2t\\
  s(t) = t \\
  y = e^{-2t}\int e^{2t} t dt\\
  y = e^{-2t}\left( \frac{1}{2} e^{2t} \cdot t - \frac{1}{4}e^{2t} + C \right)\\
  y = \frac{1}{2} t- \frac{1}{4} + C e^{-2t}
   \end{gather*}
   We see that  \(Ce^{-2t}\) is the solution to the homogeneous equation.
   \(y_p = \frac{1}{2} t- \frac{1}{4}\) is one of the solution to the nonhomogeneous equation. 
\end{example}
\begin{matlab}
  \apilink{dsolve}{https://www.mathworks.com/help/symbolic/dsolve.html}
  \begin{lstlisting}
   >> syms y(t)
   >> eqn = diff(y, t) + 2*y == t
   >> dsolve(eqn)
   ans =
      t/2 + (C1*exp(-2*t))/4 - 1/4
  \end{lstlisting}
  Note: C1 / 4 is still a constant C.
\end{matlab}